% $Id$
\labeledchapter{Iterator}

    \section{Intent}

The Iterator pattern separates the details of traversing a 
collection so that they can vary independently.

    \section{Motivation}

Iterators are much more apparent as a pattern in languages that do not
have aggregates as first class objects.  Perl has lists, along with
the variable types arrays and hashes, and it is trivial to go through
any of these with built-in Perl structures like foreach, each, map,
and so on.  The file input operator, $<>$ iterates through the lines
of the file. The readdir() and glob() functions iterate through
filenames.

In Perl we do not have to write classes to traverse an array, but we
do have to write code to go through our own complex data structures.
The Perl built-in control structures, like foreach(), map(), and grep,
do not define an interface that objects can use to control the
iteration.  They take a list, and we have to know what all of the
elements of the list at the same time. A data structure might not even
exist---the elements might be represented by an algorithm that
determines the next element so that the object does not have to store
all of the elements.

Iterators have three parts:  the {\it data}, the {\it iterator} that
defines the traversal order, and the {\it controller} which invokes
the iterator.

The logic has to know at least two things:  how to get the next
element and when no more elements are available.  Depending on the
type of traversal, it may also need to know something about its state
or the order it should follow. The controller uses this logic through
some sort of interface, which may or may not be apparent, to interact
with the data.

Since we separate each of these things when we use this pattern, we
can easily change any one of them without affecting the others since
they are {\it loosely coupled}.  We reduce their dependence on other
to give oursevles more flexibility and to improve the reusability of
our code.

For instance, we can change the traversal from depth-first to
breadth-first, and the controller says the same if it uses the same
interface.  We can use the same controller for other iterators with
the same interface, or the same interface with other controllers,
which gives us more choices if we decide we need to change something.

    \section{Types of iterators}
    
Iterators come in two types: those where the something else controls
the iteration, called {\it external iterators}, and those where
the iterator controls itself, called {internal iterators}.  Which one
you decide to use depends on what you need to do.  As the implementor
of the iterator you have to do about the same amount of work in either
case.  Other programmers benefit, or suffer, from which one you
decide to use.

    \subsection{Internal iterators}
    
With internal iterators, we tell a method or function to perform some
operation on each element of a collection.  In this case you already
know that you will have to visit most or all of the elements, and as
long as you do that you do not care how it happens.  Internal iterators
often combine the controller and iterator aspects into a single thing
which simplifies the life of the programmer who uses the iterator in his
code.

The map() and grep() built-in functions are internal iterators.  They
go through all of the elements in the data independent of our control.
We give these functions a bit of code that they apply to each of the
elements in a list.  The map() function returns a list of values based
on the original list, and the grep() function returns a list of values
from the original list that satisified a condition.  The controller
and iterator parts are part of the perl core.

\begin{quote}
\begin{verbatim}
my @squares = map { $_ * $_ } 0 .. 10;

my @odds = grep { $_ % 2 } 0 .. 100;
\end{verbatim}
\end{quote}

We could rewrite these examples with foreach(), which is a different
controller, but then we also have to add more code to build the list
we want to save.  The map() and grep() functions save us time because
they assume that we want to go through all of the elements.

When you use the File::Find module, you let its find() function
iterate through the directories on its own, and it applies the
call_back function to each file as it finds it.

\begin{quote}
\begin{verbatim}
#!/usr/bin/perl
use File::Find ();

File::Find::find( { wanted => \&wanted }, '.' );

sub wanted 
	{
    print "$File::Find::name\n" if /^.*\.tex\z/s;
	}
\end{verbatim}
\end{quote}

Modules which implement composite data structures can define
methods to iterate over all of its elements to perform an
operation, such as a call back function or a Visitor object.
The Netscape::Bookmarks module represents the information in
a Netscape bookmarks file, usually stored in HTML on a local
computer, in a data structure so you can do interesting things 
with the data such as spell-checking, re-arranging, importing
new links, converting formats, or link checking.  On the insides
it is a collection of different objects from the Netscape::Bookmarks
classes Category, Link, Separator, and Alias.


The Netscape::Bookmarks::Category module provides a recurse() function
that applies a call\_back routine to every element in the category and
below, as well as an introduce() routine that passes its elements
one-by-one to a Visitor object.  The module handles the details of 
the iteration and the controller for us.

\begin{quote}
\begin{verbatim}
use Netscape::Bookmarks;

my $bookmarks = Netscape::Bookmarks->new( $file );

$bookmarks->recurse( \&call_back );

$bookmarks->introduce( $visitor );
\end{verbatim}
\end{quote}

Classes define internal iterators to handle tasks for us when we
need to operate on every element of the collection in a uniform
matter.  All of the elements are treated the same way and when
the iterator is done with one it automatically moves on to the
next one.

    \subsection{External iterators}

You probably use external iterators all the time without even
knowing it.  When you control the iteration, you are using an
external iterator.  Since the list is a fundamental Perl
concept, Perl naturally has lots of features to work with
lists, and lots of external iterator idioms.

To go through the elements of a list, you can use the foreach()
control structure as an external iterator.  You control the
iterator because you have the ability to skip items (with next),
stop the iteration (with last).

\begin{quote}
\begin{verbatim}
foreach my $link ( $bookmarks->elements )
    {
    ...
    }
\end{verbatim}
\end{quote}
    
Some collections do not exist as lists and must be accessed
explicitly one at a time.  The while() control structure handles
the iteration by repeatedly fetching the next element to process.
In this case, you have to know how to get the next element.  The
DBI interface returns rows from a record set one at a time through
the fetchrow_array() method.

\begin{quote}
\begin{verbatim}
use DBI;

my $dbh = DBI->new(...);
my $sth = $dbh->prepare(...);

while( my @row = $sth->fetchrow_array )
    {
    ...
    }
\end{verbatim}
\end{quote}
    
The line input operator is an iterator. 
We can even mess with the black magic of the filehandle
iterface to use the line input operator to iterator over things that
are not really files, as we see later in this chapter.
    
\begin{quote}
\begin{verbatim}
while( <STDIN> )
    {
    ...
    }
\end{verbatim}
\end{quote}

Since we control external iterators, we can use more than iterator at
the same time.  If we want to compare two files, for instance,
we can use the line input operator on two different filehandles
at the same time.  Each time you go through the while loop you
get the next item from each of the iterators.

\begin{quote}
\begin{verbatim}
while( my ( $old, $new ) = (scalar <OLD>, scalar <NEW>) )
    {
    ...
    }
\end{verbatim}
\end{quote}

We also use external iterators when all of the of the data cannot or
should not be in memory at the same time. If you work with a tied DBM
hash, your hash represents possibly large numbers of keys and values
stored on disk.  Since the elements of the hash are not in memory, you
have a saving in space.  If you use the keys() or values() functions
those potentially large numbers of keys or values are now stored in
memory, negating your savings. The each() iterator fetches one
key-value pair at a time.

\begin{quote}
\begin{verbatim}
while( my($key, $value) = each %DBM )
    {
    ...
    }
\end{verbatim}
\end{quote}
        
We do not necessarily need modules and classes to create iterators
either.  We can use a closure to hold all of the information.
If we wanted to iterate over odd numbers, an infinite series
which we cannot ever completely store in memory, we can
create a closure that returns the next odd number each time we
call it.  It maintains its own state and we avoid all of the 
overhead of method lookups.  We show more on this later.

\begin{quote}
\begin{verbatim}
my $odds = do { my $next = -1;  sub { $next += 2; return $next } };

while( my $number = $odds->() )
    {
    ...
    }
\end{verbatim}
\end{quote}
    
    \section{Iterator interfaces}
    
    \subsection{Complex data structures}
    
The Netscape::Bookmarks module stores its data as a tree which is natural
since Netscape Navigator presents the collection as a tree in its interface.
The programmer who uses the module can figure out how to traverse this
structure himself (which is not very difficult), but its such a common
task that the module provides two functions to do it for you.

The recurse() function goes through the data structure depth first. For
each item the programmer can perform some sort of calculation with
a call back function.

The introduce() function does the same thing, but uses a Visitor
object instead of a callback.

    \subsection{Object methods}
    
Some modules save memory by computing elements only when needed,
or have to fetch data on request from remote sources, like our
earlier DBI example.  In these cases the object has a method
which returns the next element.


The Set::CrossProduct module lets you deal with all of the combinations of
two or more lists.  For instance, for the lists (a,b) and (1,2) you
have the combinations (a, 1), (a, 2), (b, 1), (b, 2).  The number of
combinations is (at most) the product of the number of elements in
each list, which means that the number of elements can be very large
for even a small number of moderately sized lists.  Five lists of five
items means over 3000 combinations.  Instead of storing all of those
combinations, the module simply stores the lists and keeps track of
which combination it need to make next.  It provides a next() method
that lets the external iterator fetch the next value, while we provide
the controller.

\begin{quote}
\begin{verbatim}
use Set::CrossProduct;

my $cross_product = Set::CrossProduct->new();

while( my $item = $cross_product->next )
    {
    ...
    }
\end{verbatim}
\end{quote}

\subsubsection{When is an iteration finished?}

The method interface has a problem---how do I know when no
more elements are available?  The interface has to signal to
the controller that the iterator has gone through all of the
elements.  

A false value does not always work.  The line input operator,
for instance, uses undef to signal the end of input.  Any
value besides undef comes from the data source.  I test
specifically for the undef value to see if the input is
finished.

\begin{quote}
\begin{verbatim}
while( defined( $line = <STDIN> ) )
	{
	...
	}
\end{verbatim}
\end{quote}

If the undef value is a valid value, I cannot use it to signal
the end of the iteration.  I might be able to use another value 
that does not cannot appear in the data, but if any value is
valid, I cannot use inspection to decide what to do.

I could design an iterator to have another method which tells
me the state of the iteration.  I check this method before 
I attempt to fetch the next element.  If the has\_more\_elements
method returns false, I stop the iteration.

\begin{quote}
\begin{verbatim}
while( $iterator->has_more_elements )
	{
	$item = $iterator->next;
	...;
	}
\end{verbatim}
\end{quote}

More Perl-like methods work too.  I can always return a reference to the data instead of the data
itself. If the iterator returns a non-reference, then no more elements
are available.  The interface is the almost the same as \ref{} since a
reference is always true, even if the data it points to would evaluate
to false.

If the next() method always returns a reference, perhaps an array
reference, we can tell the difference between undef and the anonymous
array of one element that contains the undef value.

\begin{quote}
\begin{verbatim}
while( $iterator->next )
	{
	$item = $iterator->next;
	...;
	}
\end{verbatim}
\end{quote}

\subsection{Iterate}

The Iterate module defines some controllers for these sorts of 
interfaces so we can interact with the object just like we do
with lists for foreach(), map(), and grep().  It defines
iterate, igrep, and imap which look almost just like the
perl built-in functions, but takes an object.

\begin{quote}
\begin{verbatim}
use Iterate qw(iterate igrep imap);

iterate { print "$_\n" } $cross_product;

my @output = imap { } $cross_product;

my @filtered = igrep { } $cross_product;
\end{verbatim}
\end{quote}
    
Each of these functions goes through all of the elements of the
object through the objects interface.  Without hints from the
object, the Iterate module uses the methods __next__ and __more__
to get the next element and determine if more elements exist.  The
object's class has to implement these methods itself, so the three
controllers work with any object that follows the interface.  The
control, iteration, and data are separate.

XXX: show the implementation of iterate

    \subsection{Closures}

Objects are data with behavior.  Closures are behaviour with data, so 
they make handy iterators.  We combine the data and iterator portion
to create the closure.

XXX: need an example

    \subsection{Tied scalars}

Tied scalars must have a FETCH method, but nothing specifies what
we have to do or which data we have to return with that method.  The
Tie::Cycle module ties an anonymous array to a scalar so that each
time you access the scalar's value, you get the next item from the 
array.  When you get to the end of the array it goes back to the
beginning.  The controller is the use of the scalar on the right-hand
side of an expression, the FETCH method defines the iterator, and the
anonymous array stores the data.

I initially created Tie::Cycle to handle alternating colors in rows
of HTML tables.  I grew weary of creating bugs when I changed the
colors or the number of colors and the amount of distracting code
that had to go into calculating an index that stayed within the
bounds of the defined elements of an array.  All I wanted was the
next color, and I wanted that to be simple.  Tie::Cycle handles
the annoying parts for me, and I can reuse it wherever I need it.

\begin{quote}
\begin{verbatim}
use Tie::Cycle;

tie my $colors, 'Tie::Cycle', [qw(aaaaaa cccccc ffffff)];

while( $sth->fetchrow_array )
    {
    my $row_color = $colors;
    
    print <<"HTML";
<tr>
    <td bgcolor="$row_color">
    ...
    <td>
</tr>
HTML
    }
\end{verbatim}
\end{quote}

    \subsection{Tied filehandles}

You can attach almost any data to a filehandle with the tie.  You
have to implement some of the functionality yourself, such as
determining what the next ``line'' is, but once you have done
that little bit of work you can use all of Perl's filehandle
iteration framework to handle the traversal.

This example ties a normal scalar to a filehandle so you
can read the scalar line-by-line or one character at a time.
The READLINE function defines how to read a line (or several
lines in list context), and the GETC function defines how
to read one character.  Perl uses READLINE when you use the
file input operator $<>$ and GETC when you use the getc()
function.  In each case, the module removes the pice that
we read, so our scalar gets shorter and shorter (unless we
add to it by some other means, which we might want to do
if we create an in-memory buffer).

We can decide how to read our lines.  In this case, we avoid
an annoying chomp by not returning the current value of the
input record separator, \$/ (a newline by default, but maybe
something different).

\begin{quote}
\begin{verbatim}
package Scalar::Iterator;

sub TIEHANDLE
    {
    my( $class, $text ) = @_;
    
    bless \$text, $class;
    }
        
sub READLINE
    {
    my $self = shift;
    
    my $EOL = $/;
    if( length $$self > 0 )
        {
        my @lines = ();
        
        while( length $$self > 0 )
            {
            $$self =~ s|(.*?)$EOL||s;
            print "Matched $1\n";
            push @lines, $1;
            last unless wantarray;
            }

        return wantarray ? @lines : $lines[0];            
        }
    else
        {    
        return;  # undef signals the end
        }
    }
    
sub GETC
    {
    my $self = shift;
    
    return $1 if $$self =~ s|(.)||s;
    
    return;  # undef signals the end
    }
\end{verbatim}
\end{quote}

We can now treat a scalar just like a filehandle.

\begin{quote}
\begin{verbatim}
use Scalar::Iterator;

my $data = ...;

tie *MY_DATA, 'Scalar::Iterator', $data;

my $line = <MY_DATA>;
print "Got one line [$line]\n";

my $char = getc( MY_DATA );
print "Got next character [$char]\n";

print "Got rest of lines:\n", <MY_DATA>;

\end{verbatim}
\end{quote}

We change the way that we traverse the scalar.  We can change
what we mean by ``'line'' and ``char'' to be something else.  After
all, the computer does not really know what these things are.  Let's
change READLINE to read the next sentence, and GETC to read the next
word.  This is more complicated than it sounds if we wanted to do this
to arbitrary text, so you have to do more work than we do for the
general case.

Ever wanted to put a regular expression into \$/?  Well, now you can.
We conveniently used \$EOL as the end-of-line marker in our example,
and we put it at the end of our regular expression in READLINE.  If
we put regular expression special characters in there, our s/// operator
will interpret them as regular expression sorts of things.  In this 
case we think we have reached the end of the sentence when we run
into the next punctuation character in the class [.!?].  This time
we include the end-of-record marker, the sentence ending punctuation,
with the sentence. At the same time we collapse all whitespace to a single
linear space.

We have to make a minor change to make GETC read words.  Let's pretend
that words only have alphabetic characters and ignore special cases
like contractions, abbreviations, and hyphenated words.  In that case,
GETC only has to return the next sequence of letters while skipping over
non-alphabetic characters it finds first.

Our data has not changed.  It still can be anything that we like, but
we can easily change how we go through it.  Since all the bits of the
iterator stay separate from the object (not really an instance, in this
case).  If we decide to change how we go through the object, the iterator
is the only thing that changes.  We can even define several different
iterators and use them at the same time if we like.  We don't have to do
much more work to turn our sentence reader into a paragraph reader, and 
so on.

\begin{quote}
\begin{verbatim}
sub READLINE
    {
    my $self = shift;
    
    my $EOL = '[.!?]';
    if( length $$self > 0 )
        {
        my @lines = ();
        
        while( length $$self > 0 )
            {
            $$self =~ s|(.*?$EOL)||s;
            my $sentence = $1;
            $sentence =~ s/\s+/ /g;
            push @lines, $sentence;
            last unless wantarray;
            }

        return wantarray ? @lines : $lines[0];            
        }
    else
        {    
        return;  # undef signals the end
        }
    }
    
sub GETC
    {
    my $self = shift;
    
    return $1 if $$self =~ s|[^a-z]([a-z])||i;
    
    return;  # undef signals the end
    }
\end{verbatim}
\end{quote}

Now that you know about tied filehandles, you should be able to imagine
infinite uses for them: return chunks of HTML or XML, etc., but beware---tied
filehandles can be very slow.  From ease-of-use, flexibility, and speed, you
only get to choose two.

    \section{Perl Modules which represent iterators}
    
Besides the modules we used as examples, several other modules express the
Iterator design pattern.

    \subsection{XML::?}
    
    \subsection{DBI}

The DBI module provides an abstract interface to databases regardless
of their type---flat-file, relational, and so on.  When we search a 
data source for matching records, the DataBase Driver (DBD) returns
a statement handle (\$sth in the DBI docs) which points to a set
of records that match.

    \subsection{File::ReadBackwards}

The File::ReadBackwards is a simple iterator that gives you the lines
from a file one at a time, only starting at the end and working its
way to the beginning. You can use its object interface or its tied
filehandle interface.  You supply the external iterator.

    \subsection{Tie::DirHandle}

The Perl built-in function readdir() is an iterator. In scalar context
it gives us the next filename from the directory handle, and in list
context give us back all of the filenames.  It defines the logic of
the traversal and we supply the iterator ( a while loop, grep, etc.).
This modules hides the readdir() behind the scenes so we can use the
filehandle iterators to interact with the directory handle.  We get
the next filename with the line input operator instead of readdir(),
which means we only have to specify the dirhandle when we tie the
handle and can forget about it otherwise.
    
    \subsection{Tie::IxHash}

Normally, hashes do not preserve the order of the elements you add to
them.  Perl stores the key-value pairs in a, well, hash tree, for easy
lookup. Several modules, including Tie::IxHash, remember the order of
hash operations, including addition of keys, so you can get the keys
back in the order that you added them.  It uses a tied hash to define
the logic of traversal, and you can then use the standard Perl (external)
iterators to traverse the hash.
