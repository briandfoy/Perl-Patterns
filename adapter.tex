% $Id$
\labeledchapter{Adapter}

    \section{Abstract}

The interface of one class translates calls to the interface
of another.

    \section{When to use an Adapter}
    
\begin{itemize}
\item You want to use a new API but need compatibility with
    legacy programs.
\item You need a different or expanded API for a legacy API
    which you cannot change
\item You need to use existing classes that have a different
    style of interface
\end{itemize}

	\section{Illustration of use}
	
Perl was still at version 4, but moving towards Perl 5 when the world
wide web was still new and people were just beginning to use dynamic
content and CGI programming.  Stephen Brenner created a Perl 4 library
named cgi-lib.pl which simplified many of the tasks that all CGI scripts
did, and as with most things that allow us to be lazy, it become very
popular, and rightfully so.  A typical cgi-lib.pl program looked like:

\begin{quote}
\begin{verbatim}
#XXX: example of cgi-lib.pl script
\end{verbatim}
\end{quote}

When Perl 5 became more popular, and its extensible, modular design
allowed Lincoln Stein to develop CGI.pm, a lot of people started using
it.  CGI.pm is completely different than cgi-lib.pl.  It has more
features, uses object oriented programming, and handles most of the
modern web eccentriticies, including \modperl.

Many people wanted to use CGI.pm since it included many bug fixes, handled
more features, and was more stable.  The cgi-lib.pl library was good
for what it did, but technology had moved on. People still had a lot
of value invested in scripts which used cgi-lib.pl.

Lincoln solved this by adding an Adapter to CGI.pm, which, with a 
couple of keystrokes, could make CGI.pm use the same interface as
cgi-lib.pl.  To use CGI.pm with a script written for cgi-lib, you
change {\tt require cgi-lib;} to {\tt use CGI qw(:cgi-lib)}.  The
previous example once converted, looks like.

\begin{quote}
\begin{verbatim}
#XXX: example of cgi-lib.pl script converted to CGI.pm
\end{verbatim}
\end{quote}


