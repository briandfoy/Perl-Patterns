% $Id$
\labeledchapter{Factory}

    \section{Abstract}

One class creates one of more instances of other classes.

    \section{When to use a Factory}
    
\begin{itemize}
\item You want to delay choosing a class until you look at the
    argument list of the constructor
\item You want to use different classes depending on the state
    of the program
\item You want to choose from among many options, some of which
    may not be available all of the time
\item You want to decide as late as possible which class to use,
	although you still want a unified interface
\item You have many polymorphic classes you might use, but you
	do not need to know which one you ended up using
\item You might use one of several classes with the same
	interface
\end{itemize}

\section{Illustration of use}

I had to create a framework for an application that made general
health assessments about the user based on personal information
that he entered.  When I started the project the application only had
a web interface, although I needed to change things around so that
the user interface was separate from the code.  Additionally, I 
had to separate the framework into five different major assessments.
Each assessment would know nothing about the interface to calculate
some health numbers, and an application could use any one of the
assessments, and possibly more assessments that would be added later.
Futhermore, some assessments might be absent based on licensing
arrangements with certain customers.

The existing code did something similar to the following code.

\begin{quote}
\begin{verbatim}

my $health;
if( $assessment eq 'Diabetes' )
	{
	require Health::Diabetes;
	$health = Health::Diabetes->new()
	}
elsif( $assessment eq 'Cardiac' )
	{
	require Health::Cardiac;
	$health = Health::Cardiac->new()
	}
elsif( $assessment eq 'Cardiac' )
	...
\end{verbatim}
\end{quote}

Notice that each assessment required explicit code to initialize
it.  A programmer would have to more explicit code to use another
assessment, or take away code to prevent the use of an assessment.

Since an application needed to be able to interact with one of
several assessments, and it did not know until it started which
assessment it could use, I decided to use a Factory.  All assessments
would start with identical interfaces, and each assessment would be
represented by its own module.  The application would be able to use
the module if it were present, but not otherwise, and the application
should not have to worry about any of this logic.

I started with five module, which I will call Diabetes.pm, Cardiac.pm,
General.pm, Wellness.pm, and Fitness.pm, but I created one module, Health.pm, 
to act as a Factory.

The application could choose an assessment through Health.pm.

\begin{quote}
\begin{verbatim}
use Health;

my $health = Health->new( assessment => 'Diabetes' );
\end{verbatim}
\end{quote}

The object that {\tt Health->new} returned is blessed into another
class, in this case Health::Diabetes.  The application does not
know this.  Health.pm documents the interface of the assessment
modules, and as long as the application follows the interface then
it does not need to know anything else about the object.

Inside the Health::new method I used some fairly simple code to
choose an assessment module.

\begin{quote}
\begin{verbatim}
package Health;

use vars qw( %Classes );

# created dynamically # XXX
%Classes = (
    DRC    => 'Health::DRC',
    FIT    => 'Health::FIT',
    CRC    => 'Health::CRC',
    HRA    => 'Health::HRA',
    GWB    => 'Health::GWB',
    );

sub new
    {
    my( $class, $args ) = @_;

    eval( "require $Classes{ $args->{assessment} }");
    croak( "Error loading " . $Classes{ $args->{assessment} } .
        "\n$@" ) if $@;

    bless $args, $Classes{ $args->{assessment} };

    return $args;
    }
\end{verbatim}
\end{quote}

Health.pm loaded the appropriate class based on its arguments,
then blessed the reference into that class.  Although I only
show the basics of what really happened, this example demonstrates
everything for a Factory.  The constructor for Health.pm returns
an object that belongs in another class based on the state
of the program.

In this example, I created the hash {\tt \%Classes} dynamically
at compile time by looking for installed modules (which, when
done once, can be cached).  As long as the assessment was 
available, 






