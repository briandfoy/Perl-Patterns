\labeledchapter{Facade}

	\section{Intent}
	
The Facade design pattern provides an easy-to-use interface to an
otherwise complicated collection of interfaces or subsystems.  It
makes things easier by hiding the details of its implementation.

	
	\section{Introduction}
	
The Facade design pattern connects the code we write for applications,
which do specific tasks, such as creating a report, and the low level
implementation that handle the details, such as reading file,
interacting with the network, and creating output.  The facade is
an interface that an application can use to get things done without
worrying about the details.  The facade decouples these layers
so that they don't depend on each other, which makes each easier
to develop, easier to use, and promotes code re-use.

I can use this design pattern to deal with a complex system that
already exists, or one that I want to make from scratch. Several
Perl modules available on the Comprehensive Perl Archive Network
(CPAN) represent facades, even if they do not admit it.

	\section{Illustration of use}
	
To request a simple file from a web site, I have to create a
connection to the web site, request the resource using a proper
HyperText Transfer Protocol (HTTP) request, receive the HTTP response,
parse the response, and finally handle the data.  I have to do much
more work if I want to handle common web features like cookies, forms,
and caching. If I want to fetch a resource from an FTP server instead
of a web server, I  need to handle a completely different protocol.

If I look at this problem, some immediate objects present themselves:
connection, request, response, and resource.  However, I only want to
fetch the resource and continue on with my real work rather than deal
with myriad objects to do something that is logically so simple. To
code this myself in a reasonable amount of time might take a couple of
screenfuls of code depending on how careful I am and how many features
I decide to support.

The LWP module (Library for WWW in Perl) provides a facade for doing
all of these things.  I tell LWP to fetch a resource and it does the
rest, including all of the protocol-specific details for HTTP, FTP, or
any other protocol that LWP understands.

In code listing \ref{LWP}, I use LWP::Simple which makes fetching a
web resource as simple as it can get.  I do not have to specify the
protocol, the connection method, or parse the response.  Indeed, if I
know the URL, and I can fetch the resource.

\begin{quote}
\begin{verbatim}
use LWP::Simple qw(get);
	
my $url = 'http://www.perl.org';
my $data = get( $url );
\end{verbatim}
\end{quote}

The LWP::Simple module is a facade---it provides a simple interface
that unifies protocol, network, and parsing aspects of the problem so
that I do one thing---fetch the resource.  If someone changes 
LWP or underlying implementations, I do not have to change my
script and I still benefit from the improvement.


\section{Focus on the task}

A facade restricts the functionality, and thus the complexity, of a
system by creating specialized interfaces for specific tasks. The
HTML::Parser module is a base class, so the programmer must write a
subclass that tells the parser what to do and when to do it, but it
does not perform a specific task other than the parsing, such as
syntax checking or data extraction.  However, as in my LWP
example, I simply want to complete a single, logical task---not program
HTML::Parser subclasses.  In reality, I like writing HTML::Parser
subclasses, but not everyone with whom I work does, so I can create
facades for them.

Facades to HTML::Parser do small, common tasks while they hide all of
the complexity behind-the-scenes.  I am the only one in the
programming team who has to understand HTML::Parser so I can create
much simpler interfaces for the rest of the team. They should not have
to write an entire subclass if they only want to extract the links of
an HTML document, for instance.  Simple things should be simple.

If I wanted to extract references (which most people call ``links'')
from an HTML document, I can use the HTML::LinkExtor module. It
handles most of the complexity for me while still giving me a
reasonable amount of flexibility through a callback mechanism. In code
listing \ref{LinkExtor}, I use a callback to extract all of the anchor
references (the HREF attribute from the A tag).  I still have to do a
bit of the dirty work since HTML::LinkExtor passes the tag name and a
list of attribute--value pairs to call\_back().  I still have to know
the details of the implementation of HTML::LinkExtor to work with it.


\begin{quote}
\begin{verbatim}
require HTML::LinkExtor;

use vars qw( @links );

sub call_back 
	{
	my( $tag, %attr ) = @_;
	return unless exists $attr{href};
	
	push @links, $attr{href};
	}
	
my $parser = HTML::LinkExtor->new( \&call_back, "http://www.example.com" );

$parser->parse_file("index.html");
\end{verbatim}
\end{quote}

I can sacrifice flexibility for convenience by using a simpler Facade,
HTML::SimpleLinkExtor, which simply returns the references but does
not have a callback mechanism.  In code listing \ref{SimpleLinkExtor},
I do the same thing that I do with HTML::LinkExtor example in code
listing \ref{LinkExtor}, and my fellow programmers do not know
how HTML::SimpleLinkExtor does it.  They do not need to worry about
writing the callback function. They simply get the result that they
need.

\begin{quote}
\begin{verbatim}
use HTML::SimpleLinkExtor;

my $extor = HTML::SimpleLinkExtor->new( "http://www.example.com" );
$extor->parse_file("index.html");

my @links = $extor->href;
\end{verbatim}
\end{quote}

If I want to do something different with HTML::LinkExtor, like
extracting URLs from tags with SRC attributes, I have
to modify the call\_back subroutine, or I can  use
a method from HTML::SimpleLinkExtor, a simpler facade, like I do in code
listing \ref{SimpleLinkExtorAll}.


\begin{quote}
\begin{verbatim}
use HTML::SimpleLinkExtor;

my $extor = HTML::SimpleLinkExtor->new( "http://www.example.com" );
$extor->parse_file("index.html");

my @all_links = $extor->links;
\end{verbatim}
\end{quote}

In both of these examples, the facades allows programmers to focus on
the task---extracting links---rather than on the programming.  Since
each facade provides a task-oriented interface, programmers do not spend time
thinking about how the task should be completed, just as they do not
think too much about HTTP or TCP/IP when they use their web browser to visit
their favorite web pages.

The HTML::SimpleLinkExtor works for most uses, but also cannot handle
complex cases at all.  Reduced flexibility is the major  consequences
of a facade. The more restrictive facades are easier to use at
the cost of flexibility. HTML::LinkExtor has more flexibility, but
is a bit more complicated and I have to do more work to use it. The
more flexible interfaces can handle more situations and respond to
special cases at the cost of simplicity.  Specific situations require
different levels of flexibility and simplicity, and as a result,
different facades, if any at all.


\section{Facades promote reusability}

Many programmers already use a sort of facade, although they typically call
it a subroutine. Subroutines did not always exist, and they represented
a pattern of their own at one time.  Today most languages take subroutines
for granted even though they are the foundation of re-usable code.

The abstract nature of the subroutine allows programmers not only to
group program statements into logical operations behind a subroutine
name, like the facades in the previous section, but they also allow
programmers to reuse that group of program statements without
repeatedly typing them.  Programmers can reuse and share collections
of subroutines that they group in libraries which can form the
interface of a facade.

What if I want to check the status of a web resource? If I went
through all of the steps myself every time I needed to do this, I
would have to create an HTTP request, connect to the web server,
receive the response, parse the response, and check the response code
against known status codes.  Later, once I have coded this four or
five lines of LWP code (or 15 to 20 lines of socket code) in all of my
applications that need it, I will probably discover that I need to fix
some of the code and to make the change in several places.  If I
put all of that code in a subroutine that all of my applications can
share, I only have to fix things in one place and programmers using
my module do not need to do anything at all.

I first ran into the link validation problem when I created a
user-configurable directory of internet resources which I wanted to
validate every day. I wanted to make sure that the links in the
directories actually led to a web page rather than the annoying ``404
Not Found'' server error.  I also wanted to catch dead links before a
customer added them to his directory.  I needed to do the same simple
task in several applications, and the few lines of repeated code in each
application seemed so innocent that I missed the obvious refactoring
potential.

With a couple of customers using the service, I did not notice
anything amiss with my validation code.  It caught dead links and did
not have false positives. With tens of users and a couple hundred
thousand resources to validate, I discovered that not all web servers
respond in the same way to certain types of HTTP requests, and that
some servers even had little known bugs.  One notorious server
returned an HTTP error for any HEAD request\footnote{ Every HTTP
request specifies a method.  A HEAD request asks the server for the
resource's meta data, but not the resource itself so it does not have
to download potentially large amounts of data.}, so I had to program
some special cases.  The task which I thought was simple grew much
more complex, but I wanted to work on the level of a ``ping''---a
simple ``yes'' or ``no'' answer. Tests on small data sets did not
reveal any problems in the parts per ten thousand range, but things
quickly got out of hand after that.

My refactored solution was a facade.  At the application level I did
not care about server eccentricities, work-arounds for HTTP
non-compliance, or most error recovery.  I simply wanted to know
if the URL actually pointed to something. 
I created a glorified subroutine, gave it a module name, and used it
whenever I needed an HTTP response code. I uploaded the module to the
Comprehensive Perl Archive Network (CPAN)
as HTTP::SimpleLinkChecker.  Code listing \ref{SimpleLinkChecker}
shows the entire facade---a single function behind which all of the
real work takes place.  The facade takes care of all of the details,
including all of my accrued knowledge about specific server behaviors,
so that it could recognize possible problems and double check
errors to a HEAD request by actually downloading the resource.

\begin{quote}
\begin{verbatim}
use HTTP::SimpleLinkChecker qw(check_link);

my $code = check_link("http://www.example.com");
\end{PerlCode}
\end{code}

If someone uses this module and decides to upgrade to a newer version when
I release one, he can get the benefit of all of my improvements and
enhancements without changing any of his code, while at the same time,
he benefits from any enhancements to its LWP infrastructure
even if he does not use the latest version of HTTP::SimpleLinkChecker.
This {\it loose coupling} makes a programmer's life much easier since
the facade hides changes to the underlying system.  Changes in other
parts of the system have little or no maintenance consequences on the
programmer's application.  Since the facade depends very loosely on
the underlying implementation, I can distribute it separately.  Other
people do not need a specific version of LWP.  I make it easy for 
people to use so that they will use it rather than going through the
pain and suffering that I did.

\section{Facades as objects}

A facade object acts as the gatekeeper of all method calls for the
underlying implementation.  Each method may in-turn act upon
additional objects or classes to perform its task, but the programmer
does not have to know anything about that. The facade object knows
which underlying objects handles the real work and delegates parts of
each task appropriately\footnote{Delegation represents another sort of
pattern with which the Facade pattern might collaborate.  Many
patterns work in concert with other patterns, and although I do not
discuss Delegation here, you can read about it in the documentation of
Damian Conway's Class::Delegation or Kurt Starsinic's
Class::Delegate}.

If I want to parse an HTML page to extract various things from the
$<$HEAD$>$ as well as some of the links, I can use HTML::HeadParser
and HTML::LinkExtor, but at the application level that is too much
detail. I am stuck thinking about HTML parsing when I should be
getting my real work done. I can also create my own HTML facade that
hides the two modules---or any other implementation---that I use.

In code listing \ref{HTML-SimpleExtractor}, I wrap the interfaces
to HTML::HeadParser and HTML::SimpleLinkExtor in a single interface
so I only have to deal with all of them in my application.

\begin{quote}
\begin{verbatim}
package HTML::SimpleExtractor;

require HTML::HeadParser;
require HTML::SimpleLinkExtor;

sub new
	{
	my( $class, $url ) = @_;
	
	return unless $data = LWP::Simple->get($url);
	
	bless \$data, $class;
	}
	
sub title
	{
	return HTML::HeadParser->new()->parse($$_[0])->header('Title');
	}
	
sub links
	{
	HTML::SimpleLinkExtor->new()->parse($$_[0])->links;
	}

1;

__END__		
\end{PerlCode}
\end{code}

Once I have my facade in place, I can write applications that I do not
need to couple to any particular module.  Since my program in code
listing \ref{gatekeeper.pl} does not know that I used HTML::HeadParser
it does not need to change if I decide to use something different for
that portion of the task. The facade hides the changes.

\begin{quote}
\begin{verbatim}
#!/usr/bin/perl -w
use strict;

use HTML::SimpleExtractor;

my $html = HTML::SimpleExtractor->new('http://www.example.com');

my $title = $html->title;
my $links = $html->links;

$" = "\n\t";

print "$title\n\t@links\n";

__END__
\end{PerlCode}
\end{code}
 
In this case, I can change any of the details involved with fetching
and parsing the HTML to something smarter and more efficient later.
This can be quite expedient when the responsibility for the facade and
the application belong to different programmers or teams, or when it
would take much longer to fully implement the facade than to finish
any of the applications.  I can create something that works today even
if it is not the best implementation then I can incrementally change
and improve it as time allows.

Even though my example is very simple-minded, I get the job done. I
hide the two modules behind the facade, and I can immediately use
HTML::SimpleExtractor in my applications. Once I have more time to
devote to the module, I can change how it does its job while all of my
applications that use it stay the same.  None of my application code
depends on the specifics of the implementation, and may not even know
that the implementation has changed.

\section{Ad hoc facades}

Most frequently the work programmers seem to do involves an
established base code that has entrenched itself into the work flow of
their organizations, and the further away they are from the creation
time of this code, the more difficult it is to maintain or learn,
especially if it is as sparsely documented as most such code I have
seen.  New team members can have an especially difficult time learning
a byzantine code base which ends up strangling the work flow.

A facade can gradually fix this without an immediate or complete
rewrite of the old code. Since a facade provides a unified interface
to an complex, underlying system, it can also hide years of
improvements, multitudes of styles, and unforeseen problems in the
original code. A new interface that connects various legacy subsystems
of the old code provides a way for programmers to replace the
functionality later while creating new applications with the new
interface. New programmers do not need to learn the entire system
if one of the old salt programmers create a facade for them.


As more and more applications use the new facade, and
hopefully fewer and fewer applications use the old code base
as programmers gradually replace them, the application base moves
towards something much more maintainable since the
application code does not rely on the underlying facade
implementation.  Programmers can re-implement portions of
that at their leisure without breaking applications.

Typically, these sorts of facades pull together a particular way of
doing things in a particular context such as a special business need
or workflow.  I might have several closely related applications, and
as I develop them in parallel parts of them start to look the same
because they do similar things and use the same resources. Such an
application's first few lines might look like code listing
\ref{lotsomodules} which uses several Perl workhorse modules.

\begin{quote}
\begin{verbatim}
#!/usr/bin/perl -w
use strict;

require cgi-lib.pl;
use DBI;
use HTML::Parser;
use HTML::TreeBuilder;
use LWP;
use Text::Template;

# my Perl implementation here
__END__
\end{PerlCode}
\end{code}

I can refactor those applications and use a facade to contain all of
the details about how I do the work. I want the facade to represent
the task, not the method. If I use certain modules by local policy,
then I only have to enforce that policy behind the facade rather than
in every application.  For this suite of hypothetical applications I
create a module I call Tsunami\footnote{a deluge of code}---the name I
give my fictional product. Instead of all of the modules I use in code
listing \ref{lotsomodules}, including the notoriously old cgi-lib.pl,
I only have to use one module, as in code listing \ref{onemodule}. 
This also means that other team members, by policy if not practice,
only use one module too. If, for some reason, policy changes so that
Tsunami should use CGI.pm instead of cgi-lib.pl, I only have to change
one file. If I improve Tsunami, everybody benefits.


\begin{quote}
\begin{verbatim}
#!/usr/bin/perl

use Tsunami;

my $wave = Tsunami->new(...);

$wave->fetch('http://www.example.com');

my $title = $wave->title();

my $txt   = $wave->as_text();

print $txt;

__END__
\end{PerlCode}
\end{code}

\section{A priori facades}

If I have the luxury of prior thought and planning, and I know
that some parts of the system resist planning, I can use a 
facade to present an application programming interface, and
build up the rest of the stuff as I find out more and more about
the problem.

Once I go through the object-oriented analysis process and have
identified the objects, I can also identify the different ways
that the programmers will use those objects.  For instance, if
I want to create an application to send messages between two 
computers, I know that I need a network object and a message
object.  These objects make it easy to deal with related sets of
information my program must maintain.


I want to create a application that can ``chat'' with another,
meaning that the two applications can send messages back and
forth between each other.  I can use a facade to represent
a simple interface, with send() and receive() methods.  I 
might need more later, but in this contrived example I pretend
that I do not know everything this application might have to
do because the specifications are fuzzy and the scope of the
problem scares the project planners (in this case, me).

When I start programming, nothing works because I have not
written any code to represent the objects, and at that level I need
all of the objects for the other ones to do their part.  Since a facade
hides these objects, it also hides their absence as well.  I can
get something in place quickly, just as in my HTML::SimpleExtractor
example in code listing \ref{HTML-SimpleExtractor}, and be on my way.

As with all new programming I start, the first thing I do is write a
test suite.  Since I have no objects to test, all of the tests should
fail, which is my first real test---tests fail when they should.  

I create my module workspace with h2xs, which creates a t directory
for test files\footnote{see Test::Harness for details about testing}.
In my test file I add the test in code listing \ref{send-test}.  Since
the send() method should die with an internal error message, I check
to see that it does. So far the module Chat.pm is the template that
h2xs created for me.

\begin{quote}
\begin{verbatim}
eval {
     Chat->send('Come here Mr. Watson, I need you');
     };
print $@ ? 'not ' : '', "ok\n";
\end{verbatim}
\end{quote}

One step beyond that I want to test that the parts of the interface
exist, so I need an interface.  I need to create the Chat.pm module.
In code listing \ref{Chat-start} I have a minimal module which has one
class method, send().  I have not really implemented it yet, so I call
the die() function if a program calls the method.

\begin{quote}
\begin{verbatim}
package Chat;

use vars qw( $UNIMPLEMENTED );
$UNIMPLEMENTED = "Not implemented!";

sub send    { die $UNIMPLEMENTED }

1;
\end{verbatim}
\end{quote}


I expect the test from code listing \ref{send-test} to fail because
send() calls the die() function, but I can modify the test to see if I
get the right message in \$@.  In code listing \ref{send-test2}, the
test succeeds even though the send() uses die(), since that is the
behavior I expect---part of the interface now exists.

\begin{quote}
\begin{verbatim}
eval {
     Chat->send('Come here Mr. Watson, I need you');
     };
print $@ eq $Chat::UNIMPLEMENTED ? '' : 'not ', "ok\n";
\end{verbatim}
\end{quote}


A similar test for an undefined method should give me a different
sort of error.  In code listing \ref{receive-test} I test to see if 
the receive() method exists, and if it does, then something is
wrong.  I have not defined receive() yet.

\begin{quote}
\begin{verbatim}
eval {
     Chat->receive('I am on my way');
     };
print $@ ? '' : 'not ', "ok\n";
\end{verbatim}
\end{quote}

Although I still do not have any objects, I can change my send()
method to take arguments.  In code listing \ref{send-args} the send()
method takes a message and a recipient argument, although I have
not said anything about what they are.  I have, however, made 
progress on the interface even though I still do not have anything
to actually do the work.

\begin{quote}
\begin{verbatim}
sub send
	{
	my( $class, $message, $recipient ) = @_;
	
	return unless defined $message and defined $recipient;
	
	return 1;
	}
\end{verbatim}
\end{quote}

My test for send() changes to make sure it does the right thing
for different argument lists.  In code listing \ref{send-args-test}
I add three tests for different numbers of arguments.  Only
the call to send() with the right number of arguments should
succeed.  The other tests check for failure when send() should
fail.

\begin{quote}
\begin{verbatim}
# should fail -- no recipient
eval {
	Chat->send('Come here Mr. Watson, I need you');
	};
defined $@ ? not_ok() : ok();

# should fail -- no message or recipient
eval {
	Chat->send();
	};
defined $@ ? not_ok() : ok();

# should succeed
eval {
	Chat->send('Come here', 'Mr. Watson');
	};
defined $@ ? not_ok() : ok();
\end{verbatim}
\end{quote}

This process continues as I add more to the interface and as I
implement the objects that will actually do the work.  In the mean
time, I have done useful work that has gotten to towards my goal, and
I have created a suite of tests to help me along the way.  The
implementation does not concern me too much at this point because I
can easily change it later.  The rest of the project depends on 
the facade. Only the facade knows about the implementation,
so everything else, including applications and tests, do not have to
wait for the complete implementation to start work.

Once I progress far enough to have Message and Recipient objects, if I
decide I need them, I can change my send() method to use them.  In
code listing \ref{check-args} I check each argument to ensure that
they belong to the proper class.  Each object automatically inherits
the isa() method from UNIVERSAL, the base class of all Perl classes,
and returns TRUE if the object is an instance of the class named as
the argument, or a class that inherits from it. All of my tests still
do the same thing.

\begin{quote}
\begin{verbatim}
sub send
	{
	my( $class, $message, $recipient ) = @_;
	
	return unless $message->isa('Chat::Message') and
		$recipient->isa('Chat::Recipient');
	
	return 1;
	}
\end{verbatim}
\end{quote}

If later I change the objects or their behavior, I have not wasted too
much time.  My facade and its tests still work. Any application I have
written does not need to change significantly.  The facade handles the
details and the interactions between the various objects. At the same
time, other programmers can start to use the interface to create
applications.  The programs will not work until everything is
complete, of course, but the programmers have a jump start on the
process because they can code and test before everything is in place. 
They essentially work in parallel, instead of serially, with the
programmers implementing the objects and the facade.  The creation of
classes is not a work flow bottleneck since the facade decouples
the application and lower level implementations.

\section{Perl modules which are facades}

Several modules exhibit a Facade design pattern, although their
authors may not have thought about design patterns or facades in
particular.  A good design stays out of the way and does not draw
attention to itself.  A good facade keeps most of us truly ignorant
about whatever is behind it.

Most of the modules on CPAN with ``Simple'' in their name use the
facade design pattern, including LWP::Simple, HTTP::SimpleLinkChecker,
and HTML::SimpleLinkExtor which I used for examples.

	\subsection{LWP}

The LWP family of modules act as a facade with a unified interface to
various network protocols including HTTP, HTTPS, FTP, NNTP, and even
the local filesystem.  It can handle some protocol specific details
too.  
	
	\subsection{DBI}
	
The DBI module provide a simple interface to several database servers
or file formats.  I can query several types of server or file formats
with the same interface while DBI---actually the appropriate
DBD---handles the connection, query, response, and other tasks.  The
DBI interface even allows me to  change the database server behind the
scenes (from SQL Server to postgresql, perhaps) without changing much
more than the DBI-$>$connect statement.  I do not need to worry about
the connection implementation, protocols, or data format.

	\subsection{Tied classes}

Perl's tie functionality is a type of facade.  What looks
like a normal Perl scalar, array, or hash stands-in for a
possibly complex object behind the scenes.  The most
impressive use of this sort of facade, in my opinion, is the
Win32::TieRegistry module, which represents the quite
complex Microsoft Windows registry as a tied hash.  This
module even sees the following as equivalent:

\begin{quote}
\begin{verbatim}
	\$Registry->{'LMachine/Software/Perl'}
	\$Registry->{'LMachine'}->{'Software'}->{'Perl'}	
\end{verbatim}
\end{quote}

I do not have to know very much, if anything, about the Registry.  I
do need to know how I name a key, but I do not need to know how it is
stored or accessed since I already understand Perl hashes and
references.  This module takes what I already know and lets me use it
instead of learning low-level vendor interfaces.

This facade also allows me to develop on foreign platforms. I can
reimplement the code so that I can have a Unix version of the Windows
registry, for instance.  This fake version of the Registry allows me
to use all of the great tools and setups I already have on my unix
accounts while targeting Windows platforms.  If my application had to
use explicit calls to the Windows programming interface I would not be
able to do that.

\section{References}

You can read more about design patterns in {\it Design Patterns},
Erich Gamma, Richard Helm, Ralph Johnson, Jon Vlissides, Addison
Wesley, 1995.

Test suites for Perl modules usually use Test::Harness, although I
find actual test files easier to understand.  I have some simple tests
in the test.pl of HTML::SimpleLinkExtractor distribution, or the t
directory of HTTP::SimpleLinkChecker.

All real modules in this article are in the Comprehensive Perl
Archive Network (CPAN) --
http://search.cpan.org
