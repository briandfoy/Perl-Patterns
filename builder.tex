% $Id$
\labeledchapter{Builder}

    \section{Abstract}

The internal representation of an object allows the object
to create different representations of the object.

    \section{When to use an Builder}
    
\begin{itemize}
\item You need to convert one format to one of many other formats.
\item The representation of the object depends on the state
of the program
\item The object might be represented in many ways.
\item You want to add new representations later without reworking
the code
\end{itemize}

	\section{Illustration of use}
	
Most web browsers have the ability to save references in a bookmarks
file so users can quickly access those references.  Most web browsers
have their own bookmarks format, but users simply care about accessing
their bookmarks no matter which browser they use, and even where
they use it.

Netscape Communications Corp. added a roaming access feature to
Navigator 4.XXX that allowed users to upload or download their bookmarks
from a central server, called an Roaming Access server. (A Perl
module on CPAN lets apache handle this).  Once the control of the 
data was passed off to the central server we could do a lot of
interesting things with the data.

I wanted to create a bookmark converter so that users could 
upload their bookmarks file and get it back in a different
format, whether that format was for another browser or their
own data format.  I created the Netscape::Bookmarks module
to parse the Navigator format (so this part was a Reader)
and create an internal representation.  Once I had the internal
representation, I created modules (the builders) to make 
new representations, such as XML, XHTML, comma separated values,
DBase files, custom HTML pages, and
relational database tables among others, and many other things.  Once the
Builder design was in place, creating a new format was simply a
matter of manipulating the internal representation.  

If I had wanted to go even farther, I could have made similar readers
for other formats that converted them to the same internal representation.
Then, any format could be converted to any other. Since Navigator was the
only browser that would upload bookmarks files, I never got that far.

Jon Bentley's Programming Pearls (notice anything similar?) includes
an article about the internal representation problem.  Suppose we had
N formats, which we wanted to convert to any of the other formats.
If we had explicit algorithms to go directly from one format to 
another, then we would have to have N(N-1) algorithms.  If everything
could be converted to a common, intermediate format, then we only
need N algorithms.  
