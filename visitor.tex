% $Id$
\labeledchapter{Visitor}

    \section{Abstract}

The Visitor patterns provides an easy way for me to
separate complex data structures from the operations I
want to perform on those structures.  Other users can add
new operations without affecting the modules that implement
access to the data structure, which allows more flexibility
and promotes reusability.

	\section{Perl Modules which use a visitor}
	
Netscape::Bookmarks
MacOSX::iTunes
Data::Grove

   \section{Motivation}

When I create something useful, people tend to use it for
everything other than what I thought would be useful.  Some
of my modules are really glorified interfaces to deeply nested
hashes---Netscape::Bookmarks, Mac::iTunes, and Mac::PropertyList,
for example.  I created these modules because I needed to deal
with the data in a Perl program in some particular way.  With
Netscape::Bookmarks, I wanted to export database information
to the format the Navigator (and now Mozilla) could use.
I created Mac::iTunes and Mac::PropertyList because I wanted
to programmatically change my iTunes configuration.  I have
a 40 gigabyte hard drive bulging with MP3 files that I want
to add to iTunes, but I do not want to add all of the files
individually.  Perl can do all of these things for me if I
can represent the data somehow.

Once I release these modules, people use them for tasks that
I never considered.  I do not want to add every transformation
or task to the module anytime someone comes up with a new use.
If I provide a way for external modules to interact with my
data structures, I virtually eliminate any coupling between
operations and data.  My module represents the data, and that
is the end of it.

   \section{Illustration of use}

The module that implements the data structure can provide a
way for an external module, the I<visitor>, to traverse its
data structure and interact with each object. The objects do
not have to be of the same type or have the same interface. 
The module simply guarantees that the visitor has a chance
to see each object. The visitor class defines operations for
the data structure, and these operations may change the data
structure or not.

The Netscape::Bookmarks modules represents a bookmarks file
for Netscape Navigator or Mozilla as a tree structure behind
the scenes---a deeply nested hash of hashes. The programmer
does not need to know how I implemented the data structure
as long as they follow the published model interface.  I can
change things behind-the-scenes as long as the interface
gives the same results.

This tree structure can be quite deep, so simple operations
can incur a lot of extra setup work.  The simple operation
has to traverse the data structure and make decisions about
what to do at each step.  The Netscape::Bookmarks data
structure is heterogenous---it is a nested collection of
four other types of objects, one of which is another
container. A task like link checking
can take quite a bit of extra work just to set up the
controls to handle the details of the data structure which
ultimately is implementation dependant.  Any operations
which I do not define in the module has to muck with
the internal data structure directly.  

That is, new operations would have to do this work unless I
provided a general way for the module to handle the overhead
automatically.  The Visitor design pattern provides a way to
handle this situation.

The Visitor pattern separates the operations from the
data.  It is the object oriented version of a callback.
One module defines access the data, and along
with that it defines a way to traverse the data.  The
traversal mechanism exposes one internal object at a 
time.  The external module, the visitor, then operates
on the presented object.  When the visitor is done with
one object, the data module presents another one. 



Netscape::Bookmarks::Category has an introduce()
function that takes an object as an argument. This object is
the visitor and must have a visit() method.  Since
Netscape::Bookmarks::Category contains all of the
information about the data structure, it
should be the part that knows how to traverse it.  As
introduce() steps through the data structure, it calls the
visitor() method on each object with the Visitor object as
its argument.  Once ``inside'' the object, the node object
turns around and calls the visit() method on the Visitor
object with itself as the argument.  The Visitor object then
determines what sort of object it has and what it wants to
do with it.

XXX: introduce code

XXX: method code



It may seem that there is an extra level of redirection
here. Why not use the Visitor's visit() method directly in
introduce? The extra layer, the visitor() call, gives each
object a chance to accept or decline to be visited, perform
checks on the visitor object to ensure it conforms to a
specification, or inspect the results of the Visitor object.
 The element may wish to change its state after the visitor
does it's work.  For instance, a Netscape::Bookmarks::Link
object may wish to update its LAST\_VISIT field (which
refers to the last time I selected the link, not anything to
do with this pattern). Some objects may be private or for
internal use only, so they would decline a visitor
(basically a no-op). The visitor object should not be
responsible for these things, or have to break
encapsulation. All the visitor knows is the published
interface.

The Netscape::Bookmarks distributions comes with several
visitor examples.  The natural one is a link checker.  The
bookmarks file has categories, links, separators, and
aliases, and I only need to visit link objects.  I create a
visitor class which has a method for each sort of object I
wish to visit.  I instaniate a visitor object, and call the
introduce() function.  The Category object handles the
traversal, and each time it encounter a link, the visitor
checks the link.  The visitor class is very small since it
only defines the operations I want to perform, although I
can as many operations in a single class if I like.

XXX: link check visitor

Any other time I want to do something with each element, I
can create another visitor class.  The original classes do
not change.  People who do not need a particular visitor
operation do not need the class which defines them, and
other programmers can create new functionality without
dealing with the original code base which is kept in a
stable state.  Modules which define operations are not
strongly coupled to the data---they can be distributed
independently.

	\section{Callbacks}
	
The non-object implementation of the Visitor pattern is the
callback, typically used by things which traverse or iterate
through a bunch of things which might be objects but
probably are not.  Each time the function encounters a token,
thingy, element, or whatnot, the thingy is passed to the
function.  This way, the code providing the thingys does not
need to foresee every possible operation that the programmer
may want to do.  It simply supplies the way to see each
object.

The File::Find module, which comes with Perl, is a Perl
replacement for the find command line tool.  The
programmer supplies a callback function to the find
function.  The find function handles all of the 
transversal logic.  The easiest way to start with
File::Find is find2perl which turns find commands into
Perl scripts.  Code listing XXX shows a slightly edited
version of the output from \texttt{find2perl . -name "*.pm" -print}.
The find function applies the wanted function to each
file it finds.  The 

\begin{quote}    
\begin{verbatim}
#!/usr/bin/perl

use File::Find ();

File::Find::find({wanted => \&wanted}, '.');
exit;


sub wanted {
    /^.*\.pm\z/s;
    print("$File::Find::name\n");
}
\end{verbatim}
\end{quote}    
	
\begin{quote}    
\begin{verbatim}
	XXX: HTML::Parser
\end{verbatim}
\end{quote}    

	\section{Possible implementations}
	
	\subsection{one method per class}

A lot of other literature on the Visitor pattern has a
method for each sort of object.  In the Netscape::Bookmarks
module, there are four sorts of object XXX, so the class
would look like:

\begin{quote}    
\begin{verbatim}
XXX: code
\end{verbatim}
\end{quote}    


	
	\subsection{dispatch tables}

Some objects may have the same interfaces, so you want to
reuse methods for some of them.  You still check the
object's type, but then use a lookup table to decide which
method to use.

\begin{quote}    
\begin{verbatim}
XXX: hash with keys of class names and values of method names
\end{verbatim}
\end{quote}    

Once you look up the method name, you can call it

\begin{quote}    
\begin{verbatim}
	$self->$method($object);
\end{verbatim}
\end{quote}    
	
	\subsection{AUTOLOAD tricks}
	
You might be able to use a single method for all of the
object processing. This can be especially handy when all of
the objects have the same interface, or share a common
method name (polymorphism), or you don't need to call any
methods on the object (!!).  Suppose we have a collection of
heterogeneous objects that each have a title method, and we
want to count the number of objects at the same time we
print put its type and title.

\begin{quote}    
\begin{verbatim}
sub AUTOLOAD
	{
	my $self   = shift;
	my $object = shift;
	my $method = $AUTOLOAD;
	
	$count{$AUTOLOAD}++;
	
	print "$AUTOLOAD=> ", $object->title, "\n";
	}
\end{verbatim}
\end{quote}    
		
Suppose we wanted to turn on debugging output for objects in
the collections that supported it:

\begin{quote}    
\begin{verbatim}
sub AUTOLOAD
	{
	my $self   = shift;
	my $object = shift;

	$object->debug('on') if $object->can('debug');
	}
\end{verbatim}
\end{quote}    

	\section{Designing classes for Visitors}
	
Besides the infrastructure to allow visitors to interact
with objects, you can also add other things to your objects
to make your life easier.  Since you are expecting someone
to iterate through your collection with a visitor, you can
supply polymorphic methods that do the right thing for each
object.


		\subsection{String representation}
		
You supply a method to turn the object into a string
representation, no matter the object.  You might be
especially lazy and use Data::Dumper, but each method still
has an as\_string method.  The visitor may decide not to
call as\_string on every object, but it could if it wanted
to. A programmer may wish to call as\_string on objects
which meet certain criteria.
		
		\subsection{Debugging or Tracing}
		
		
		\subsection{Peeking}
		
		
		\subsection{Update}
		
The observer Pattern describes a way to notice several
objects that something has happened. The visitor can be the
way that it communicates with each object.
	
