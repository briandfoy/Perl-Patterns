% $Id$
\labeledchapter{Memoized (Modified) Singleton}

    \section{Abstract}

For the same arguments, the constructor returns the same instance.

    \section{When to memoize}
    
You may consider using a Singleton class if your code
have any of these symptoms:

\begin{itemize}
\item You get an identical instance given the same arguments,
but different instances for different arguments
\item You try to use the singleton pattern, and it gets you
most of the way there, but not all the way
\end{itemize}

    \section{Motivation}
    
The singleton pattern allowed us global access through a single
instance, although that instance could have different names in 
the program.  The Memoized singleton pattern modifies that so
the same list of arguments to the method that returns that
instance gets the same instance, while a different list may
return a wholly different instance.  The pure singleton solved
our sharing problems with something that exists singly like a
printer or the syslog.  The Memoized singleton shares things that
may exist multiply, like configuration files or output filehandles,
which should be shared if two parts of our program want to share
the same one.

    \section{A simple example}
    
Once we have read and parsed a configuration file, we should not
have to do it again, although if we need to use it in different
parts of the program, we should not have to do too much work
to do that.  It sounds like we should use a singleton to represent
the configuration file.  If we used a pure singleton in which the 
class can only have one instance, then we can only have one configuration
file.  What if we need to look at multiple files, like .profile and .bash\_profile
when we login to a bash shell, or httpd.conf, access.conf, and srm.conf
from apache?  A pure singleton will not allow us to represent each of
these at the same time from the same class, although they represent the
same concept.

What we want is only one instance for each configuration file so that
if we have already parsed the file, we skip all of that work and 
simply access the information.  To do this, we need to remember which 
files we have already seen. We can modify are previous singleton
examples to use a hash instead of a scalar to remember our instances.
They key of the hash will be the name of the configuration file, and
its value will be the configuration file instance that goes with it.

Let's override the new() method from ConfigReader::Simple, available
on CPAN, to do this.  Our memoized version of new() turns ConfigReader::Simple
into a singleton class.

\begin{quote}    
\begin{verbatim}
package Config::Singleton;

use base qw(ConfigReader::Simple);

my \%Configs = ();

sub new
    {
    my( $class, $file, @args ) = @_;
    
    return $Configs{$file} if exists $Configs{$file};
    
    my $object = $class->SUPER::new( $file );
    return $object unless ref $object;
    
    $Configs{$file} = $object
    }
\end{verbatim}
\end{quote}
    
\todo{EXPLAIN CODE HERE}

    \section{Connecting to multiple databases}
    
In our pure singleton example we shared a single database connection.
The first time that the program asked to connect to the database our
connect() method passed the argument list on to the DBI->connect()
method, then saved the returned database handle if the connect() 
succeed.  Each subsequent time we tried to connect we got a reference
to the same database connection even if we had asked to connect to
a different database.  We had simply ignored the arguments altogether.
However, its completely reasonable  to connect to multiple
databases, or the same database multiply, in certain situations.  I have
frequently kept two database handles active as I read from one to write
to the other.

We need to modify our connect() method in the DBI::Singleton package and we have 
to modify our \$Dbh to store multiple database handles.  However, this
situation is much more complicated than our single argument example.  How
do we recognize that two argument lists are the same?  We cannot use
a list of the key for a hash.  We could serialize the list so that
we end up with a string, perhaps with something like

\begin{quote}
\begin{verbatim}
my $key = join "\0", @_;
\end{verbatim}
\end{quote}
    
but what if the character "\\0" is a valid within the data?  It is improbable
that we will experience a collision with this technique, but it is certainly
possible.  How would be differentiate between the list ( "a\\0", "b" ) and
( "a", "\\0b" ) ?

    
For most of the cases in which we would select a Memoized Singleton, the
Memoize module available on CPAN will do all of the hard work for us.  It will
digest the argument lists and store the return values for each.  We
simply tell Memoize which methods to affect, then move on to the next
programming problem  In this case we only
have to memoize the constructor, which is a class method (so the first
argument is always the class name).     

    \section{Full example with database connections}
    
\todo{Full example code}

    \section{Apache::DBI}
    
Before mod\_perl, one of the problems with web databases was that each
CGI script needed its own connection to a database, and setting up
that connection could be the major bottleneck in the program making the
web site seem slow.  The mod\_perl extension to Apache solved many of
the performance problems with CGI scripts (which you can read about elsewhere),
but you still had to connect to the database.  Since mod\_perl processes stayed
in memory between web requests, everything that needed a database connection
maintained an open database connection, causing resource use problems.

The mod\_perl folks solved that problem with Apache::DBI which is an 
example of a memoized singleton.  For each unique list of arguments,
Apache::DBI stores one database connection that all parts of the 
process can share.  It even pings the database for you and re-establishes
the connection if need be.  You might already use Apache::DBI without 
knowing it since DBI looks for it when it is loaded, and will forward
connection requests to Apache::DBI if it is already loaded.
