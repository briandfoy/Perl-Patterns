% $Id$
\labeledchapter{Perl Features that make this possible}

We assume that you are already familiar with object oriented 
programming in Perl, or at least have some experience with it
in some other language, and that you have access to the Perl
documentation that comes with Perl.

All of the patterns that we discuss in this book are built
as modules, since part of the pattern design process is 
building reuseable components.  A lot of the Perl features
and techniques that we use to create a pattern revolve
around features for module creation or classes that are
typically used by modules, such as Exporter and AutoLoader.
You may already be familiar with the conventional ways of
using these features, but they have more power than that
if we think about them just a little differently than usual.
Indeed, one of the greatest compliments to any design is its
usefulness in ways never imagined.

    \section{How to use this chapter}

This chapter is a reference for the design patterns that we
discuss in the second half of this book.  If you are already 
familiar with object-oriented programming in Perl and the concepts
that you see in the section headings for this chapter then you
may want to skip ahead.  Otherwise, you may want to refer to this
chapter when you do not understand a technique or trick that we
use in a pattern.  If you are doing some sort of penance, then
you should read this entire chapter before going on to the next.

        \subsection{Use the Perl documentation.}

For more details on any of the concepts we discuss, you can get
more details by looking at the Perl documentation.  Every standard distribution
of Perl comes with documentation in a form appropriate for that
operating system.  You may also want to start using Perldoc.com which
has indexed and made searchable the Perl manual pages and a lot of the
documentation for popular modules.  If you have a web browser and access
to the internet then you can always get to the documentation.

Aside from the documentation, the definitive reference is {\it Programming Perl}.

% $Id$

    \section{UNIVERSAL}

Every Perl object is ultimately derived from a special base class
named UNIVERSAL, which is the same sort of thing as Object in
SmallTalk or java.lang.Object in Java.  Every class and object
automatically inherits from UNIVERSAL even though we do not have to
explicitly state this inheritance.  The UNIVERSAL package is the root
of every inheritance tree.

\begin{center}
XXX: image ?
\end{center}

This package defines several methods that can be used on any object or
class to do minimal introspection on those objects.  We discover
various properties of classes and objects through the methods defined
in UNIVERSAL, as well as apply our own ideas of objects by extending
it.
        
        \subsection{can}

The can() instance method takes a method name as an argument and
returns TRUE if that object knows it can call that method, and FALSE
otherwise. It uses perl's internal method lookup code.

\begin{quote}
\begin{verbatim}
my $input = LWP::Simple->new;

if( $input->can( 'head' ) )
    {
    # do something interesting.
    }
\end{verbatim}
\end{quote}

If can() returns FALSE, that does not necessarily mean that the object
cannot call this method---only that it does not know that it can.  See
the discussion of the Autoloader for some examples of these ``hidden''
methods.

We may use this method to figure out which objects we want to affect. 
Suppose that we want to print any objects that can be printed as a
string.

\begin{quote}
\begin{verbatim}
my @printable_objects = grep { $_->can( 'as_string' ) } @objects;

foreach( @printable_objects }
    {
    print "$_\n", $_->as_string, "\n";
    }
\end{verbatim}
\end{quote}

        \subsection{isa}

The isa() instance method takes a package name as an argument and
returns TRUE if the object is blessed into that class or any
superclass of it, and FALSE otherwise.  This is a lot better than
simply checking the object type with ref().

	my $class = ref $object;
	
The ref() function only tells you which class the object is blessed
into.  If you use it to check if the object can do a certain task or
should be used in a certain way, you need to check more than its
class.  The example in the Banyard Object-Oriented tutorial (the
perlboot man page) creates an Animal class, and then Horse, Cow,
and Sheep classes which inherit from animals.  If we need to feed
all of the animals, we do not care which concrete class they are in,
only that they are Animals.

\begin{quote}    
\begin{verbatim}
foreach my $object ( @objects )
	{
	$object->eat if $object->isa('Animal');
	}
\end{verbatim}
\end{quote}
	
We also do not have to know about changes to the types of Animals
we might have.  If we want to feed all of the Sheep, we might
say

\begin{quote}    
\begin{verbatim}
foreach my $object ( @objects )
	{
	$object->eat if $object->isa('Sheep');
	}
\end{verbatim}
\end{quote}

and even if we have different sorts of Sheep subclasses (BlackSheep,
Lambs, Rams, Ewes), all of the Sheep still get to eat.  That is,
we do not have to know about all of the sub-classes ahead of time.
We only have to know that the subclass comes from the class that
we care about, and since it inherits from that class we expect it
to act the same (or close it).  The programmer can add subclasses
without our permission.

You can also use a special two argument form of isa() to check the
variable type to which an unblessed reference  (that is, a reference
that is not an object) points.  This is close the \todo{what do i need
to write here} indirect object notation where the first argument to
the method is the object itself.  In this case you cannot call methods
on unblessed references so this notation is the only one you can use.

\begin{quote}    
\begin{verbatim}
my $array_ref = [ qw( 1 2 3 ) ];

if( UNIVERSAL::isa( $array_ref, 'ARRAY' ) )
    {
    print "@$array_ref\n";
    }
\end{verbatim}
\end{quote}


For example, if you had a collection of objects and you wanted to extract
all of the objects that are of a particular type, say array references, you
could use isa() as a grep() condition.

\begin{quote}    
\begin{verbatim}

my @arrays = grep { UNIVERSAL::isa( $_, 'ARRAY' ) } @objects;

\end{verbatim}
\end{quote}

We use this technique when one object contains or controls many other
objects and wants to affect all of its subservient objects of a particular
type, such as in the patterns \todo{insert pattern names}.

\begin{quote}    
\begin{verbatim}
foreach( grep { UNIVERSAL::isa( $_, 'Foo' ) } @objects )
    {
    $_->update unless $_->status;
    }
\end{verbatim}
\end{quote}

The CGI.pm module uses this technique to discover if some of its
methods received a FILEHANDLE or a GLOB in its argument list.

\begin{quote}    
\begin{verbatim}
sub to_filehandle {
    my $thingy = shift;
    return undef unless $thingy;  
    return $thingy if UNIVERSAL::isa($thingy,'GLOB');
    return $thingy if UNIVERSAL::isa($thingy,'FileHandle');
    if (!ref($thingy)) {  
        my $caller = 1;
        while (my $package = caller($caller++)) {
            my($tmp) = $thingy=~/[\':]/ ? $thingy : "$package\:\:$thingy";
            return $tmp if defined(fileno($tmp));
        }     
    }         
    return undef;
}
\end{verbatim}
\end{quote}

        \subsection{Getting the class name}

If you need to get the class name for an object, you can use the
perl built-in ref() function.

        \subsection{Testing an instance}
        
You may want to test an reference to see if it is an instance of a 
class.  All objects inherit from the UNIVERSAL package so you
only need to check to see if the object has UNIVERSAL in its
ISA chain.  Since you cannot call methods on references, you
cannot use the method form of isa().  You have to use the
function version instead.

\begin{quote}
\begin{verbatim}
	my $is_object = UNIVERSAL::isa( $reference, 'UNIVERSAL' );
\end{verbatim}
\end{quote}

If you like, you can import the functions from UNIVERSAL:

\begin{quote}
\begin{verbatim}
	use UNIVERSAL qw( isa can );
	
	my $is_object = isa( $reference, 'UNIVERSAL' );
\end{verbatim}
\end{quote}
	

        \subsection{Adding methods to UNIVERSAL}
        
We can directly affect any class we want because Perl is not a bondage
and discipline language.  Although most of the uses we have thought up
for this, one very useful trick is to add methods to UNIVERSAL so that
we can interact with all objects in a particular way.  We have to be
very careful when we do this though since subclasses may redefine the
method that we create.

Before perl5.004, the UNIVERSAL package had two convenience methods,
class() and is\_instance().  The is\_instance function did the same
thing that we did when we tested a reference to see if it was an
instance.  We can add these functions anywhere in our code, so
you should not modify the UNIVERSAL.pm file itself---not everyone
has your modified version of the standard module, and you have to
re-edit the file everytime you upgrade.

The trick of these functions is the difference in behavior of
regular references and objects.  You cannot call methods on
references, and you do not want your code to blow up if you
do.  The is\_instance() and class() functions should tell
the difference between the two and do the right thing.  Once
you code it, you should not have to code it again.

The is\_instance() function only returns true if the object
is an instance.  We can safely assume that package names will
not be '0', the only non-trivial false value.  If you do have 
a package name '0', you have more problems than this book can
solve.

\begin{quote}
\begin{verbatim}
sub UNIVERSAL::is_instance;
	{
	my $arg = shift;
	
	return unless ref $arg;
	
	return UNIVERSAL::isa( $arg, 'UNIVERSAL' );
	}
\end{verbatim}
\end{quote}


The class() function returns the class name of the object, or FALSE if 
the argument is not an object.  If we used the ref() function, we
would get a TRUE value for any reference whether or not the thingy
was blessed into a class.  In this case, we want to assume all of
those are false values without having to enumerate all the sorts 
of references (test yourself to see if you can to find out how
many you would miss)\footnote{ SCALAR, ARRAY, HASH, CODE, REF, GLOB,
Regexp }.


\begin{quote}
\begin{verbatim}
sub UNIVERSAL::class
	{
	my $arg = shift;
	
	return unless UNIVERSAL::is_instance( $arg );
	
	return ref $class;
	}
\end{verbatim}
\end{quote}

We can do the same thing in two steps if we like:

\begin{quote}
\begin{verbatim}
my $class = UNIVERSAL::is_instance( $arg ) ? ref $arg : '';
\end{verbatim}
\end{quote}
	
but you show your intent and spend less typing with

\begin{quote}
\begin{verbatim}
my $class = UNIVERSAL::class( $arg );
\end{verbatim}
\end{quote}

We can also add much more interesting methods to UNIVERSAL.  If
we want every object to have a debug() method, even if the 
author of the module did not provide one, we add one to UNIVERSAL
(although not to the UNIVERSAL.pm file).

\begin{quote}
\begin{verbatim}
sub UNIVERSAL::debug
    {
    require Data::Dumper;
    
    my $self = shift;
    
    my $class = ref $self;
    
    print "-" x 50", Object $class from caller foo\n";
    print Dump( $self );
    }
\end{verbatim}
\end{quote}    

Once we have created this method, all objects can call it.  If an object
is an instance of a class that already has a debug method, then it will use 
that method since Perl finds it first as it traverses the ISA tree.
No matter which object we have, we can now always say

\begin{quote}
\begin{verbatim}
$object->debug
\end{verbatim}
\end{quote}    

 


     \section{References}
     
     
     \subsection{Comparing references}

Have you ever accidently de-referenced an object, or even an ordinary
reference, while printing a string?  You probably got some funny looking
output that looked like the object's package name and what you might
want to use as a memory address.  Although there is no way in Perl to
take that stringified      

    \section{Inheritance}

Inheritance defines a familial relationship between classes, which may
have parents and children (or base and derived) classes.  Base classes
can define behavior for the objects that are derived from them, just
like all classes can use the behavior defined in UNIVERSAL.  Perl keeps
track of this using a special package array @ISA (pronounced ''is a'', as
in ''a camel is a mammal'').  Base class names are placed in the @ISA array.
Those base classes might also have their own base classes.  To complicate
matters even more, a class can inherit from more than one class at a time.
This leads to all sorts of problems unless we know that the multiple base
classes will not collide.  Several of the patterns use multiple inheritance,
but in a completely safe way.  The whole of the inheritance relationships
is a tree, with the UNIVERSAL class at its root.

\todo{nifty LaTeX figure of a tree}

When Perl cannot find a method in the derived class, it starts a depth first
"climb" (since it goes towards the root of the tree, which is actually
at the top) of the inheritance tree.  It first looks in the first class named 
in the current @ISA.  If it does not find the method in that class, it looks
in the first class named in that classes @ISA, so forth, working its way
back down and up if there is multiple inheritance.

\todo{example traversal}

    \section{Subclassing}

We can extend or modify a class by subclassing the base class.

    \subsection{SUPER}

The SUPER class is a special class name which connotes the immediate
parent class of the object.  It works on a per-object basis rather
than thinking about the current package or the package in which
it was defined.  This feature, which some people think is a 
mis-feature, is really quite handy for design patterns.  We can create
objects, re-bless them, pass them around, store them in complex structures,
and all other manner of chaos and SUPER still means the object's
parent class.

    \section{bless}

Perl is not an object oriented language, although it can do a very
good job pretending to be one.  Depending on your personal philosophy 
of language design, you probably either would give your right hand
for this or want to take Larry's right hand for making it that way.

Perl knows what class an instance thinks it is because we tag
the object with a string that is the class name.  That is really
all that bless does.  An object is really just a blessed reference,
or more simply, a reference that has a name tag.

\begin{quote}    
\begin{verbatim}
my $self = \%hash;

bless $self, $class
\end{verbatim}
\end{quote}

To read the name tag of an object, we can use ref() with which returns
the string bless() tagged it.  We could also use the UNIVERSAL::isa
function to check to see if the name tag is a particular string, as
we saw before, but we can also ref().

\begin{quote}    
\begin{verbatim}
my $scalar = 'foo';
bless $scalar, time;

my $time_created = ref $scalar;
\end{verbatim}
\end{quote}

There is absolutely no requirement that the class actually exist,
or that we use anything from the class with the tagged variable.
Indeed, one of the authors is perverse enough to use this tag
as out-of-band meta-information about the variable.  Consider
a variable

    \section{AutoLoader}

The most common use of Autoloader involves breaking up a module
with AutoSplit then loading methods only when they are needed. 
Large modules, such as \todo{some module names}, inherit from AutoLoader and put
their methods after the \_\_END\_\_ token.  When the module is
autosplit, each method is put into its own file in a special
directory.  At run-time, when an object calls a method which is
not known, the AUTOLOAD routine in the AutoLoader package is
called.  This routine looks into the special auto directory for
a file with the right name to match the method name and package
for the requested method.  If it is found it is loaded and the
program continues.

Autoloader only does this by default, however.  We can create our
own AUTOLOAD routine to do just about anything that we want.

One disadvantage of this approach is that there is presently no
way to re-enter the inheritance tree search process once we
have fallen into an AUTOLOAD routine in any of the packages we
were searching.  We either have to do something useful or give
up.  Rumor has it that the next version of Perl will be more
understanding.


    \section{Operator overloading}

We can change the way that operators behave with certain objects
by redefining what they do, or overloading them.  You probably
already recognize the common operators such as X, but may not
know about the operators XXX.  We can redefine the behavior
of each of these to suit our needs.


When we redefine an operator we change its behavior for all
instances in a class.  


    \section{Tied variables}

Perl has a nifty feature that allows us to use ordinary looking 
variables in the ordinary way but redefine what they actually do.
For instance, with a scalar variable we can assign to it, access
its value, interpolate it, and so on.  With tie we tell Perl that
these normal operations are actually going to be carried out by
a hidden object from a package that we create.  This package has 
special, predefined method names to carry out the operations of the
variable.



    \section{Comparison to other languages}

Despite what you may have been taught, there is a difference
between object oriented programming and object oriented languages.
This is not the typical assertion that an object oriented language
is a bondage-and-discipline language.  Indeed, C++ is a very strict 
language, but it is not an object oriented one even though you use
it to create classes and objects.  
 
        \subsection{C++}

C++ sucks for a number of reasons and we endeavour to list all of 
them here.

        \subsection{Java}

Java sucks less than C++, but suffers from an overzealous marketing
department.  

        \subsection{Python}

Python is a language for people who just want to do what they are
told and go home at the end of the day.        
        
        \subsection{Smalltalk}

Smalltalk is for geeks not satisfied with how things work and who
are brazen enough to change the class structure themselves.  They
distribute these changes as "Goodies" like some pervert giving 
tainted candy to halloween youngsters.

        \subsection{Eiffel}

Its named after something French.  I do not like France.

        \subsection{Ruby}

If you want to work in Perl for the rest of your life, you have to
kill Ruby which might be the next big thing.

        
