% $Id$

    \section{UNIVERSAL}

Every Perl object is ultimately derived from a special base class
named UNIVERSAL, which is the same sort of thing as Object in
SmallTalk or java.lang.Object in Java.  Every class and object
automatically inherits from UNIVERSAL even though we do not have to
explicitly state this inheritance.  The UNIVERSAL package is the root
of every inheritance tree.

\begin{center}
XXX: image ?
\end{center}

This package defines several methods that can be used on any object or
class to do minimal introspection on those objects.  We discover
various properties of classes and objects through the methods defined
in UNIVERSAL, as well as apply our own ideas of objects by extending
it.
        
        \subsection{can}

The can() instance method takes a method name as an argument and
returns TRUE if that object knows it can call that method, and FALSE
otherwise. It uses perl's internal method lookup code.

\begin{quote}
\begin{verbatim}
my $input = LWP::Simple->new;

if( $input->can( 'head' ) )
    {
    # do something interesting.
    }
\end{verbatim}
\end{quote}

If can() returns FALSE, that does not necessarily mean that the object
cannot call this method---only that it does not know that it can.  See
the discussion of the Autoloader for some examples of these ``hidden''
methods.

We may use this method to figure out which objects we want to affect. 
Suppose that we want to print any objects that can be printed as a
string.

\begin{quote}
\begin{verbatim}
my @printable_objects = grep { $_->can( 'as_string' ) } @objects;

foreach( @printable_objects }
    {
    print "$_\n", $_->as_string, "\n";
    }
\end{verbatim}
\end{quote}

        \subsection{isa}

The isa() instance method takes a package name as an argument and
returns TRUE if the object is blessed into that class or any
superclass of it, and FALSE otherwise.  This is a lot better than
simply checking the object type with ref().

	my $class = ref $object;
	
The ref() function only tells you which class the object is blessed
into.  If you use it to check if the object can do a certain task or
should be used in a certain way, you need to check more than its
class.  The example in the Banyard Object-Oriented tutorial (the
perlboot man page) creates an Animal class, and then Horse, Cow,
and Sheep classes which inherit from animals.  If we need to feed
all of the animals, we do not care which concrete class they are in,
only that they are Animals.

\begin{quote}    
\begin{verbatim}
foreach my $object ( @objects )
	{
	$object->eat if $object->isa('Animal');
	}
\end{verbatim}
\end{quote}
	
We also do not have to know about changes to the types of Animals
we might have.  If we want to feed all of the Sheep, we might
say

\begin{quote}    
\begin{verbatim}
foreach my $object ( @objects )
	{
	$object->eat if $object->isa('Sheep');
	}
\end{verbatim}
\end{quote}

and even if we have different sorts of Sheep subclasses (BlackSheep,
Lambs, Rams, Ewes), all of the Sheep still get to eat.  That is,
we do not have to know about all of the sub-classes ahead of time.
We only have to know that the subclass comes from the class that
we care about, and since it inherits from that class we expect it
to act the same (or close it).  The programmer can add subclasses
without our permission.

You can also use a special two argument form of isa() to check the
variable type to which an unblessed reference  (that is, a reference
that is not an object) points.  This is close the \todo{what do i need
to write here} indirect object notation where the first argument to
the method is the object itself.  In this case you cannot call methods
on unblessed references so this notation is the only one you can use.

\begin{quote}    
\begin{verbatim}
my $array_ref = [ qw( 1 2 3 ) ];

if( UNIVERSAL::isa( $array_ref, 'ARRAY' ) )
    {
    print "@$array_ref\n";
    }
\end{verbatim}
\end{quote}


For example, if you had a collection of objects and you wanted to extract
all of the objects that are of a particular type, say array references, you
could use isa() as a grep() condition.

\begin{quote}    
\begin{verbatim}

my @arrays = grep { UNIVERSAL::isa( $_, 'ARRAY' ) } @objects;

\end{verbatim}
\end{quote}

We use this technique when one object contains or controls many other
objects and wants to affect all of its subservient objects of a particular
type, such as in the patterns \todo{insert pattern names}.

\begin{quote}    
\begin{verbatim}
foreach( grep { UNIVERSAL::isa( $_, 'Foo' ) } @objects )
    {
    $_->update unless $_->status;
    }
\end{verbatim}
\end{quote}

The CGI.pm module uses this technique to discover if some of its
methods received a FILEHANDLE or a GLOB in its argument list.

\begin{quote}    
\begin{verbatim}
sub to_filehandle {
    my $thingy = shift;
    return undef unless $thingy;  
    return $thingy if UNIVERSAL::isa($thingy,'GLOB');
    return $thingy if UNIVERSAL::isa($thingy,'FileHandle');
    if (!ref($thingy)) {  
        my $caller = 1;
        while (my $package = caller($caller++)) {
            my($tmp) = $thingy=~/[\':]/ ? $thingy : "$package\:\:$thingy";
            return $tmp if defined(fileno($tmp));
        }     
    }         
    return undef;
}
\end{verbatim}
\end{quote}

        \subsection{Getting the class name}

If you need to get the class name for an object, you can use the
perl built-in ref() function.

        \subsection{Testing an instance}
        
You may want to test an reference to see if it is an instance of a 
class.  All objects inherit from the UNIVERSAL package so you
only need to check to see if the object has UNIVERSAL in its
ISA chain.  Since you cannot call methods on references, you
cannot use the method form of isa().  You have to use the
function version instead.

\begin{quote}
\begin{verbatim}
	my $is_object = UNIVERSAL::isa( $reference, 'UNIVERSAL' );
\end{verbatim}
\end{quote}

If you like, you can import the functions from UNIVERSAL:

\begin{quote}
\begin{verbatim}
	use UNIVERSAL qw( isa can );
	
	my $is_object = isa( $reference, 'UNIVERSAL' );
\end{verbatim}
\end{quote}
	

        \subsection{Adding methods to UNIVERSAL}
        
We can directly affect any class we want because Perl is not a bondage
and discipline language.  Although most of the uses we have thought up
for this, one very useful trick is to add methods to UNIVERSAL so that
we can interact with all objects in a particular way.  We have to be
very careful when we do this though since subclasses may redefine the
method that we create.

Before perl5.004, the UNIVERSAL package had two convenience methods,
class() and is\_instance().  The is\_instance function did the same
thing that we did when we tested a reference to see if it was an
instance.  We can add these functions anywhere in our code, so
you should not modify the UNIVERSAL.pm file itself---not everyone
has your modified version of the standard module, and you have to
re-edit the file everytime you upgrade.

The trick of these functions is the difference in behavior of
regular references and objects.  You cannot call methods on
references, and you do not want your code to blow up if you
do.  The is\_instance() and class() functions should tell
the difference between the two and do the right thing.  Once
you code it, you should not have to code it again.

The is\_instance() function only returns true if the object
is an instance.  We can safely assume that package names will
not be '0', the only non-trivial false value.  If you do have 
a package name '0', you have more problems than this book can
solve.

\begin{quote}
\begin{verbatim}
sub UNIVERSAL::is_instance;
	{
	my $arg = shift;
	
	return unless ref $arg;
	
	return UNIVERSAL::isa( $arg, 'UNIVERSAL' );
	}
\end{verbatim}
\end{quote}


The class() function returns the class name of the object, or FALSE if 
the argument is not an object.  If we used the ref() function, we
would get a TRUE value for any reference whether or not the thingy
was blessed into a class.  In this case, we want to assume all of
those are false values without having to enumerate all the sorts 
of references (test yourself to see if you can to find out how
many you would miss)\footnote{ SCALAR, ARRAY, HASH, CODE, REF, GLOB,
Regexp }.


\begin{quote}
\begin{verbatim}
sub UNIVERSAL::class
	{
	my $arg = shift;
	
	return unless UNIVERSAL::is_instance( $arg );
	
	return ref $class;
	}
\end{verbatim}
\end{quote}

We can do the same thing in two steps if we like:

\begin{quote}
\begin{verbatim}
my $class = UNIVERSAL::is_instance( $arg ) ? ref $arg : '';
\end{verbatim}
\end{quote}
	
but you show your intent and spend less typing with

\begin{quote}
\begin{verbatim}
my $class = UNIVERSAL::class( $arg );
\end{verbatim}
\end{quote}

We can also add much more interesting methods to UNIVERSAL.  If
we want every object to have a debug() method, even if the 
author of the module did not provide one, we add one to UNIVERSAL
(although not to the UNIVERSAL.pm file).

\begin{quote}
\begin{verbatim}
sub UNIVERSAL::debug
    {
    require Data::Dumper;
    
    my $self = shift;
    
    my $class = ref $self;
    
    print "-" x 50", Object $class from caller foo\n";
    print Dump( $self );
    }
\end{verbatim}
\end{quote}    

Once we have created this method, all objects can call it.  If an object
is an instance of a class that already has a debug method, then it will use 
that method since Perl finds it first as it traverses the ISA tree.
No matter which object we have, we can now always say

\begin{quote}
\begin{verbatim}
$object->debug
\end{verbatim}
\end{quote}    

 
