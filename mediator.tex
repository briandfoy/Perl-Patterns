% $Id$
\labeledchapter{Mediator}

    \section{Abstract}

A mediator class controls the communication between objects so that each 
object does not need to know the details about which objects exist or 
what their interface supports.  The mediator object knows which methods
each object supports, and has a way to choose between objects when more
than one object can respond to the same method.

    \section{Perl modules which implement mediation}
    
\begin{itemize}
\item Foo
\item Bar
\end{itemize}

    \section{Motivation}

Large systems may need to communicate with other parts of the system.
In a dynamic object oriented system various jobs within the system, and
to increase the flexibility of the system we do not want to tie the functionality
to a particular object.  The mediator decides which object should handle the
method based on what we tell it about the system.  For instance, we once created
a system that stored configuration information in a database, and within the system
there was a configuration object and a database object.  The database did a lot
more than the configuration portion, although we could avoid the configuration
object if we wanted, although the configuration object ultimately used the database
class.  This turned out to be a major maintenance headache in the system which was
really around a hundred different separate scripts that could, in some cases, act
independently of each other.  Every one of these scripts started by including the
configuration module and the database module, meaning that any time we wanted to
change how we included those modules we had to change all of the scripts.  Once
we moved to a mediator design, each independent script only had to use the mediator
object 

    \section{Illustration of use}

A medium-sized system that I created provided a framework from which 
developers could write programs to support a web application whether
those programs were CGI scripts, \modperl modules, or command-line
utilities.  Most of the scripts would access a configuration file,
a database server, and perhaps several other resources.  The beginnings
of each script looked something like this simplified code:

\begin{quote}    
\begin{verbatim}
use Product::Database;
use Product::Config;

my $config   = Product::Config->new( $file );
my $database = Product::Database->new( $config->database_server );
\end{verbatim}
\end{quote}    

This structure worked as long as I did not have to change anything
about the Product::Config or Product::Database modules.  This sounds
more harmless than it actually was.  Since I did not start with a good
design, I could not refactor or improve the framework without making
changes in lots of other places, even with multi-file search and replace.
Any time you have to make changes in
several different places for one logical change, you should consider that
a better design could help you.  In this case, where I was neck deep in 
deadlines, production schedules, and a system that worked by was becoming harder
and harder to maintain, I could not see a solution until Kurt Starsinic 
took one look at it and said ``have you ever heard about delegation?''.  He
immediately made me read appropriate sections in { \it Design Patterns }, which 
led me to the Mediator pattern.  

I reduced the above code in each script to a couple of lines.

\begin{quote}    
\begin{verbatim}
use Product;

my $product = Product->new( $config_file );

\end{verbatim}
\end{quote}    

This small shift in design had a major maintainability benefit since it
decoupled the scripts from the details.  Each script dealt with the
generic Product object, which in turn handled all of the details of the
framework. I could change those details without disturbing other code.

The Product object had Product::Config, Product::Database,
Product::Error, Product::WebUserAgent, and other objects that the script
did not see and did not need to know about. Instead of dealing with
multiple objects, the scripts uses a single point of contact, the
Product object deals with one object.  Each module could shed a lot of
common code, such as that for error handling, since it could be handled
in the Product module, which also unified supporting code like error
handling.

Additionally, this design simplified the framework code significantly.
Some of the modules, like Product::Config and Product::WebUserAgent
needed functionality from Product::Database since it provided access
to all persistent information.  The single point of communication applied
to the contained objects as well.  The Product module decorated (see 
Pattern XXX) each object so that it could communicate with the Product
object through a method.

\begin{quote}
\begin{verbatim}
$self->parent->method_name
\end{verbatim}
\end{quote}

The Product module received the call to method\_name and would handle
it appropriately.  The Product object could easily communicate with
all of its contained objects by stepping through whatever data
structure it used to hold them.

    \section{How the design works behind the scenes}

One object acts as a single point of communication for all of the objects 
that it controls.  Once it Mediator object receives a method, it can handle
it in many ways.  Remember that the Mediator is a design, not an implementation.
Behind the scenes the Mediator may use Delegation, Chain of Responsibility,
or something else to do the job.  You may hard code behavior directly into
the module, or do something more dynamic.

	\section{Unrelated objects}
	
	
	\section{Related objects}
	
Some or all of the objects contained in the Mediator may depend on
each, in which case we call them colleagues.  Without the Mediator,
each colleague must explicity refer to each other colleague and know
about the details of each of them.  Each colleague is tightly coupled
to each other colleague.  Any change in any colleague ripples through
the entire system and vice versa.  Code that appears to be modular because
it uses namespaces, packages, and objects is actually monolithic and
difficult to separate. Consider the Product::Config module which needs
to get information from the database server.  A tightly coupled implementation 
refers to Product::Database explicitly.

\begin{quote}    
\begin{verbatim}
XXX: code that shows Config and Database coupling
\end{verbatim}
\end{quote}    

Since Product::Config uses Product::Database explicitly, it might
also use a separate database connection (unless Product::Database uses
the Singleton pattern).

With the Mediator, the objects can communicate with each other without
knowing about the other objects.  The mediator can provide a way for
each colleague object to communicate with the Mediator by decorating
each of the objects (see the Decorator pattern) so that it can refer to
it's mediator.  The colleague class doesn't even need to know which
Mediator class it uses.

\begin{quote}    
\begin{verbatim}
	my $mediator = $self->mediator;
	
	my $info = $mediator->database_call;
\end{verbatim}
\end{quote}    

More simply, we can skip the intermediate variable.

\begin{quote}    
\begin{verbatim}
	my $info = $self->mediator->database_call;
\end{verbatim}
\end{quote}    



When a Product::Config
object wants to get information from a database server it asks the Mediator.

\begin{quote}    
\begin{verbatim}
$self->mediator->database_call
\end{verbatim}
\end{quote}    

