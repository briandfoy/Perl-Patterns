% $Id$
\labeledchapter{Iterator}

    \section{Intent}

The Flyweight creates a few, fine grained objects that can be
shared so that they can be used many times efficiently.
 
    \section{Motivation}
    
Suppose we create things which share the same data over and over
again.  We shouldn't have to store the duplicated data. A Flyweight
stores the data once, and let's everyone share it. Further, in
situations where this might happen once, it probably happens quite a
bit, so you can create a Flyweight Pool that holds several of these
shareable objects.

	\section{Illustration of use}

I like to watch movies---a lot of people do, and i also like to keep
track of movies I've seen or purchased so I can .... I keep track of
screenwriters, directors, and cinematographers for each movie.  If I
wanted to represent a movie as an object, I could create an object
that contains other objects in a HAS-A relationship.  A movie has a
director, a cinematographer, and some screenwriters and producers.

\begin{quote}
\begin{verbatim}
	Movie
		Director
		Screenwriter
		Cinematographer
		Producer
		Actor
\end{verbatim}
\end{quote}
		

Each one of these is a person though and there is really no difference
between them.  I should only ever have to have one object that
represents John Cusack, for instance, even though he might be the
director, producer, or the screenwriter, or even more than one of
those in the same movie.

Without a flyweight, if i wanted to make a Movie instance for High
Fidelity, I would have to make separate John Cusack objects for
Producer, Writer, and Actor.  I have duplicated the same data three
times.  What if I make instances for the other John Cusack movies that
I like (and there are many)?  I would have to create even more John
Cusack objects.  Multiply this further by all the people in all of the
movies I care to track.

The flyweight solves the problem by sharing objects that can be used
in different contexts in the same time.  The object itself stores the
minimum data it needs (called its intrinsic state)---the stuff that
does not change depending on the context.  John Cusack's name,
birthday, height and so on do not change depending on which role in
the movie. Some things do depend on the context (called the extrinsic
state)---his age at the time he worked on the movie, for instance.  We
can calculate his age based on the context (the particular movie), but
be do not need to store it in the object.

Let's create a Person class that acts as a flyweight class:


\begin{quote}
\begin{verbatim}
	XXX: create class
\end{verbatim}
\end{quote}
		
	
The class creates singleton objects with a factory.  Since every actor
should have a unique name (according to the Screen Actor's Guild), we
can get a Person object by specifying their name.  The first time
around, we create an instance and add it to a pool that we store in
the class data.  The next time we ask for an object for the same
person, we get the same object back.  The two references point to the
same object.  When we ask for an object for another person, the same
thing happens and another object is added to the flyweight pool.  The
savings in memory are easy to see even on the back of the envelope. 
If I created movie instances for John Cusack's filmography, 52 films,
in which he played as many as 4 roles in the production of
them---about 60 uses of a John Cusack object so far. If I want to add
actor objects for Joan Cusack, his sister, I have another 10 or so
objects.  With a flyweight pool, I only use two objects, one for each
actor---a 30 times reduction.

	\section{Using context}
	
What if the object needs to know it's context (what we called
extrinsic state earlier)?  Pass that information to the object through
a method call.  Let's calculate the age of actors for the years they
were in movies. We store their birthday because we magically fetched
it from a database (or something like the IMDB).  The movie adds the
temporal context, which we pass to a new method:

\begin{quote}
\begin{verbatim}
	sub age
		{
		my( $self, $year ) = @_;
		
		# calc year
		
		return $year;
		}
\end{verbatim}
\end{quote}
		
	\section{Immutability}

Most of the times, the data in the flyweight instance will not 
change.  Since it does not store state, no other part of the
program should need to change it.

However, if it does change, everyone who shares the object benefits
from the change.  If I mistyped an actor's name, or discovered an
error in their information, such as an incorrect birthday, I update
it in one place. 

	
	\section{Another example}
	
You can implement the flyweight pattern in a couple of more ways.
Let's start with a bunch of MP3 files.  If we ripped these files
from the same CD, the files share some common information---artist,
genre, album---and we should not have to store this information
more than one time in any of our data structures.  Every MP3 file
can share a reference to the same data.


Extend this to several albums.

Once we have a bunch of MP3 files, we want to organize them into
playlists.  Each MP3 object can store the location of the file
along with some information about the file and its contents.  The
playlists are collections of references to MP3 objects.

XXX:

The MacOSX::iTunes module uses this a flyweight pool because it
doesn't want to duplicate a lot of data, but also because the iTunes
Music Library file does the same thing.  The data structure in MacOSX::iTunes
is a collection of MacOSX::iTunes::Playlist objects, which in turn are 
collections of MacOSX::iTunes::Item objects.  The MacOSX::iTunes::Item objects
represent a flyweight pool.  The module only stores the Item information
once even though it can belong to multiple playlists.  I do not have to
keep a separate object for each XXX song each though it appears it my
XXX, XXX, XXX, XXX, and XXX playlists.

The module identifies items by filename, although it might also use things
like the MD5 digest, a combination of MP3 meta data.
